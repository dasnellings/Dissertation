\abstract

Vascular malformations are a diverse class of focal lesions that may occur throughout the body and affect different vascular beds. Since the advent of next-generation sequencing it has become clear that virtually all vascular malformations are caused by postzygotic genetic changes occurring in a single cell: somatic mutations. 

Somatic mutations are widely accepted to be the initial, catalyzing event for vascular malformation. These mutations are a mix of gain-of-function and loss-of-function and occur in canonical vascular genes, or known oncogenes. From the past two decades of research, we have identified numerous genes which contribute to vascular malformation; yet, despite this genetic diversity, the mutations identified in individual malformations has thus far been---without exception---monogenic.

In this document I present the first type of vascular malformation where digenic somatic mutations are a common and critical component of lesion pathogenesis. Furthermore, I propose that somatic mutations cause not only lesion initiation, but may drive the initiation, progression, and predisposition to vascular malformation. 

In support of this hypothesis, I first identify that vascular malformations in hereditary hemorrhagic telangiectasia are initiated by somatic mutations via a two-hit mechanism. Second, I determine that cerebral cavernous malformations harbor up to 3 distinct somatic mutations that synergize to fuel lesion progression. Finally, I show that developmental venous anomalies harbor a somatic mutation which creates a mosaic field of mutant cells that predisposes to cerebral cavernous malformations. 