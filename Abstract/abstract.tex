\abstract

In this document, I will attempt to convince you that \textbf{somatic mutations drive the initiation, progression, and predisposition to vascular malformations}. The role of somatic mutations in the initiation of vascular malformations is well documented across the spectrum of vascular anomalies. However, here I present the first evidence of secondary mutations in the vascular malformation `cerebral cavernous malformation' (CCM) that drives progression and predisposition. While CCM is currently the only vascular malformation In addition, I show that vascular malformations associated with hereditary hemorrhagic telangiectasia (HHT) form via an established genetic two-hit mechanism, and explore the effects of distinct activating mutations in \italicize{GNAQ} including a mutation that causes Sturge-Weber syndrome (SWS). 

Vascular malformations are a diverse class of focal lesions that may occur throughout the body and affect different vascular beds (e.g. arteries, veins, capillaries). Distinct from cancer, vascular malformations remain functional structures and lack the capacity for metastasis and uncontrolled growth. Despite their differences, vascular malformations and cancer share an underlying pathogenic mechanism: somatic mutations. The majority of somatic mutations in vascular malformations cause gain of function (GOF) in genes involved in vascular development and/or general proliferation (e.g. \italicize{TEK}, \italicize{PIK3CA}, \italicize{KRAS}). Conversely, some vascular malformations such as CCM and CM-AVM are caused by loss of function (LOF) somatic mutations. 