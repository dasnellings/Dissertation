\abbreviations

% You can put here what you like, but here's an example
%Note the use of starred section commands here to produce proper division
%headers without bad '0.1' numbers or entries into the Table of Contents.
%Using the {\verb \begin{symbollist} } environment ensures that entries are
%properly spaced.

%\section*{Symbols}
%\begin{symbollist}
%	\item[$\mathbb{X}$] A blackboard bold $X$.  Neat.
	% Optional item argument makes the symbol/abbr
%	\item[$\mathcal{X}$] A caligraphic $X$.  Neat.
%	\item[$\mathfrak{X}$] A fraktur $X$.  Neat.
%	\item[$\mathbf{X}$] A boldface $X$.
%	\item[$\mathsf{X}$] A sans-serif $X$. Bad notation.
%	\item[$\mathrm{X}$] A roman $X$.
%\end{symbollist}

\section*{Abbreviations}

\begin{symbollist}
	\item[VM] Vascular Malformation
	\item[CCM] Cerebral Cavernous Malformation
	\item[HHT] Hereditary Hemorrhagic Telangiectasia
	\item[AVM] Arteriovenous Malformation
	\item[DVA] Developmental Venous Anomaly
	\item[SWS] Sturge-Weber Syndrome
	\item[PWS] Port-Wine Stain
	\item[FFPE] Formalin-Fixed Paraffin-Embedded
	\item[COSMIC] Catalog Of Somatic Mutations In Cancer
	\item[LOH] Loss Of Heterozygosity
\end{symbollist}
