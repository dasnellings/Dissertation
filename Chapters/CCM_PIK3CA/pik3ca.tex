\chapter{\italicize{PIK3CA} Mutations Fuel Cerebral Cavernous Malformation Growth}
\label{chap:pik3ca}

\blfootnote{This chapter is adapted from a study published in \italicize{Nature} (Ren \& Snellings et al., 2021)}
\clearpage

\section{Introduction}
Vascular malformations such as cerebral cavernous malformations (CCMs) that arise in the central nervous system are an important cause of stroke and disability in younger individuals \citep{heiskanen1993, fischer2013}.  Most CCMs arise sporadically as single lesions, but a minority present as part of a familial, autosomal dominant form of the disease that is associated with multiple lesions \citep{cavalcanti2012}. Classic genetic studies have associated familial CCM disease with heterozygous germline loss of function mutations in three genes, KRIT1, CCM2, and PDCD10, that encode the components of a heterotrimeric protein complex (the “CCM complex”) \citep{fisher2014, plummer2005}. Subsequent studies have demonstrated that CCM lesions harbor an additional somatic mutation in the same gene as the germline mutation, implicating biallelic loss of function of the affected CCM gene as the cause of the disease \citep{gault2005, akers2009}. Consistent with this monogenic loss of function mechanism, sporadic CCMs harbor biallelic somatic mutations in one of the CCM genes, resulting in homozygous loss of function \citep{mcdonald2014}. Mouse models confirm that deletion of any of the CCM genes in the brain endothelial cells of neonatal mice confers CCM lesions, proving a causal role for loss of CCM complex function in this disease.  These studies have generated the current genetic model of CCM pathogenesis in which biallelic loss of function mutations in a single CCM gene is sufficient for CCM lesion development. 

Recent genetic, biochemical, and cellular studies have revealed that loss of CCM function results in vascular lesion formation due to increased MEKK3-KLF2/4 signaling in brain endothelial cells \citep{cullere2015, cuttano2015, zhou2016, renz2015}, and that CCM disease may be strongly influenced by the gut microbiome due to its role in stimulating brain endothelial TLR4 receptors that signal through MEKK3 \citep{tang2017, mcdonald2014, tang2019}. A genome wide association study (GWAS) to identify genetic modifiers of CCM disease identified polymorphisms in the genes encoding TLR4 and its co-receptor CD14 that up-regulate their expression and accelerate disease course \citep{choquet2014, tang2017}, but other genetic drivers of lesion genesis or growth have not been identified or suspected.

Serial imaging studies to define the natural history of human CCMs has revealed that most are slow-growing and clinically silent \citep{akers2017, alshanisalman2012, horne2016}. In contrast, those that cause stroke and seizure are typically fast-growing and associated with repeated lesional hemorrhage \citep{awad2019, porter1997}. Such aggressive, symptomatic lesions are surgically resected if possible to prevent or treat associated neurologic complications, but surgery is associated with high morbidity and cost and is impractical for patients with multiple lesions or lesions in less reachable locations such as the spinal cord. Why a subset of CCM lesions exhibits rapid growth associated with clinical symptoms is unknown. Recent mouse and human studies suggest that a gut microbiome containing more invasive gram negative bacteria or an impairment of the gut barrier that blocks translocation of bacterial products such as lipopolysaccharide may modulate CCM growth through effects on TLR4-MEKK3-KLF2/4 signaling in brain endothelial cells \citep{tang2019, tang2017, polster2020}. Plasma biomarkers of angiogenesis have also been correlated with lesional clinical activity \citep{girard2018, lyne2019}. However, individuals with familial CCM disease who harbor numerous silent lesions identified by MRI imaging also often manifest symptomatic hemorrhage and aggressive growth of a single lesion \citep{polster2019}. Thus the current understanding of the environmental and genetic factors that contribute to CCM growth fails to explain important aspects of the disease natural history, especially the emergence of rapidly growing symptomatic lesions which account for the majority of clinically significant outcomes.

Recent studies of sporadic vascular malformations have identified acquired gain of function mutations in a number of central signaling pathways, including the RAS/MAPK/ERK pathway in congenital hemangiomas and capillary malformations and the PI3K/AKT/mTOR pathway in venous and lymphatic malformations \citep{tenbroek2019, rodriguezlaguna2019, castillo2019, wetzelstrong2017, luks2015, limaye2015} (and reviewed in \citep{queisser2018}). Many of these gain of function mutations, e.g. those in PIK3CA, the catalytic subunit of PI3K, are identical to those that have been identified in cancer cells \citep{dastillo2016, castel2016, limaye2015, koren2015, samuels2005}. However, unlike cancer, in which mutations in multiple driver genes such as tumor suppressor genes and oncogenes combine to promote growth \citep{bailey2018, mcgranahan2015}, the pathogenesis of vascular malformations has been considered monogenic. The studies described below reveal that symptomatic CCM disease arises through a cancer-like paradigm in which the accumulation of multiple somatic mutations in the same cell results in both the loss of a vascular malformation suppressor gene (i.e. the CCM gene) and the gain of vascular malformation growth gene (i.e. PIK3CA). 
	
In the present study we have used mouse genetic models and studies of human CCM lesions to more fully define the drivers of this vascular malformation. We find that CCM formation requires a high level of PI3K/mTOR activity in addition to loss of CCM function and gain of MEKK3-KLF2/4 signaling.  In the mouse, loss of endothelial CCM function alone is sufficient to drive CCM formation in the neonatal brain, while adult loss confers CCM formation in the testis but not the brain, a pattern that associates the site of lesion formation with the underlying rate of endothelial proliferation. Genetic studies demonstrate that PIK3CA gain of function synergizes strongly with CCM loss of function in the neonatal mouse brain, and that both are required to generate CCMs in the adult mouse brain. In humans, bulk DNA sequencing of clinically symptomatic CCM lesions that were surgically resected identified PIK3CA gain of function mutations in a majority of lesions in addition to CCM gene loss of function mutations, including several cases in which no fewer than three distinct somatic mutations were identified. Single-nucleus analysis of human CCM lesions further revealed that mutations in CCM genes and PIK3CA were most often present in the same cell, direct evidence of an endothelial cell autonomous “triple hit” mechanism. Molecular studies to identify how loss of CCM function and gain of PIK3CA function intersect revealed no change in KLF2/4 expression with gain of PIK3CA function, but identified increased phospho-AKT and phospho-S6 with loss of CCM function or following over-expression of the MEKK3 effector KLF4, consistent with cell autonomous augmentation of PI3K-mTOR signaling by CCM deficiency and gain of MEKK3-KLF2/4 function. Forced expression of ADAMTS5, a second KLF2/4 effector recently demonstrated to participate in CCM lesion formation in vivo \citep{hong2020}, also synergized with PIK3CA gain of function during CCM formation but did not detectably alter phospho-AKT or phospho-S6. Consistent with these findings, Rapamycin treatment to block mTORC1 signaling was highly effective in preventing CCM formation driven either by CCM loss of function alone in neonatal mice or by combined CCM loss of function and PIK3CA gain of function in adult mice. These studies reveal that clinically significant cavernous malformations arise through a compound genetic mechanism like that previously described for cancer, identify PI3K signaling as a major downstream effector pathway in CCM disease, and suggest that symptomatic CCMs may be effectively treated with the approved drug Sirolimus. 


\section{Results}
\subsection{\italicize{PIK3CA} Mutations Occur in Familial and Sporadic CCMs}
\subsection{CCMs Harbor Multiple Somatic Mutations in Different Genes}
\subsection{\italicize{PIK3CA} and CCM/\italicize{MAP3K3} Mutations in the Same Cell}
\subsection{Developmental Venous Anomalies Predispose to Malformation}

\section{Discussion}
\subsection{Three-Hit Model of CCM Pathogenesis}
\subsection{Similarities to the Genetic Mechanism of Cancer}
\subsection{Role of Clonal Expansion in Mutagenesis}
\subsection{Therapeutic Implications}
\subsection{Distinct Properties of \italicize{PIK3CA} vs. CCM/\italicize{MAP3K3} Mutations}
\subsection{DVA Predispose to CCM and Other PI3K-Related Diseases}

\section{Methods}
\subsubsection{CCM Collection}
\subsubsection{Brain AVM Collection}
\subsubsection{DNA Extraction}
\subsubsection{Droplet Digital PCR}
\subsubsection{SNaPshot}
\subsubsection{Sequencing}
\subsubsection{Sequence Analysis}
\subsubsection{Single-Nucleus DNA Sequencing}
\subsubsection{Statistics}

\section{Contributions and Acknowledgements}