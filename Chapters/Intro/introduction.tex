\chapter{Introduction}
\label{chap:intro}

\clearpage

\section{Vascular Malformations}
Vascular malformations (VMs) are a class of lesion that consist of malformed---but functional---blood vessels. VMs may occur throughout the body and may affect various vascular beds (i.e.~capillaries, veins, and arteries). While VMs may be restricted to a single type of vessel, they are often observed as mixed vascular malformations (e.g.~arterio-venous malformation) and mixed vascular-lymphatic malformations (e.g.~capillary-lymphatic-venous malformation). In the past VMs were considered to be congenital lesions; however, while many VMs are present at birth, there are numerous examples of spontaneous \italicize{de novo} VM formation in adulthood. 

VMs virtually always occur as focal lesions, affecting one region of the vasculature, rather than presenting as a systemic vascular defect. This observation likely reflects the fact that a systemic defect in vascular development is likely incompatible with life, therefore individuals with such severe vascular defects do not survive to birth. 

There is strong genetic component to VMs and many inherited VM disorders that initially appear to contradict the focal nature of VMs. In people with an inherited VM disorder, the pathogenic mutation is present in every cell in their body. Why then, do such individuals develop a VM in their left hand and not their right hand? The answer is that additional local events occur after birth that seed the formation of a VM in a particular location; many such events in VMs take the form of somatic mutations. 

\section{Genetic Mechanisms of Vascular Malformation}
VMs may occur sporadically in otherwise healthy individuals, however there are also several familial mendelian disorders characterized by the development of VMs. Many of the familial VM disorders are inherited via an autosomal dominant allele. Despite the systemic presence of the causal mutation in these disorders, the resulting VMs are strictly focal lesions