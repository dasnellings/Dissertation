\chapter{Introduction}
\label{chap:intro}

\clearpage

\section{Overview and Significance}
The growth and remodeling of blood vessels is a tightly regulated process that must be maintained from early embryogenesis until death. When this process goes awry blood vessels can become malformed leading to altered, and sometimes pathological function. These malformed blood vessels are termed `Vascular malformations' (VMs). VMs may occur anywhere throughout the body, but virtually always occur as focal lesions---affecting one region of the vasculature, rather than presenting as a systemic vascular defect. Even though VMs are focally restricted, they are a source of significant morbidity and mortality in affected individuals. The clinical phenotypes produced by VMs vary substantially, though they are generally prone to rupture (due to increased flow through a fragile vascular bed) or hemorrhage (due to leaky endothelial junctions). The VMs that I will discuss in the following chapters commonly form in the brain and may cause a range of neurological phenotypes including: stroke, epilepsy, seizures, and neurological deficit.

Understanding molecular pathogenesis of VMs is critical for developing methods of treatment and prevention. Although VMs exhibit massive phenotypic heterogeneity, it is becoming increasingly clear that many---if not all--VMs form as a result of somatic mutations.

Somatic mutations are well established as a critical, initiating factor in VM development; however, there remain many types of VM for which the genetic etiology is poorly understood. My dissertation seeks to understand the contribution of somatic mutations to the genetic etiology of Hereditary Hemorrhagic Telangiectasia. Furthermore, I seek to identify novel mechanisms by which somatic mutations fuel the initiation, progression, and predisposition to Cerebral Cavernous Malformations. 

In Chapter 2, I determine that VMs associated with Hereditary Hemorrhagic Telangiectasia (HHT) form via a genetic two-hit mechanism whereby a somatic mutation in \italicize{ENG} or \italicize{ACVRL1} results in biallelic loss of function. HHT is a mendelian disorder inherited by an autosomal dominant allele containing a mutation either \italicize{ENG}, \italicize{ACVRL1}, or \italicize{SMAD4}. Previous studies established that these inherited mutations result in loss of function. This finding lead to the predominant theory that the VMs associated with HHT are the result of haploinsufficiency. However, this theory is inconsistent with the presentation of VMs in HHT. The VMs associated with HHT present as focal lesions, rather than manifesting as a systemic vascular defect as might be expected by haploinsufficiency. This disconnect suggests that a local event is necessary for the development of VMs in HHT. I tested whether this local event may be a somatic mutation in the single remaining wild-type allele of the mutated gene. Such a mutation would cause in biallelic loss of function and result in the complete loss of functional gene product in the cells harboring the somatic mutation. I sequenced 19 mucosal telangiectasia (a cutaneous VM) from individuals with HHT and found a loss of function somatic mutation in 9 of 19 telangiectasia. Each somatic mutation I identified occurred in the same gene as the respective inherited mutation. Using long-read sequencing I showed that pairs of somatic and germline mutations exist in a \italicize{trans} configuration, confirming that these mutations resulted in biallelic loss of function. Furthermore I showed that multiple telangiectasia from the same individual harbored unique somatic mutations, suggesting that each telangiectasia is initiated by a different mutational event rather than a metastasis-like mechanism. 

In Chapter 3, I identify a novel somatic mutation in Cerebral Cavernous Malformations (CCMs) that occurs in the \italicize{same cell} as additional somatic mutations that synergize to fuel CCM progression. Similar to HHT, CCM is a mendelian disorder inherited by an autosomal dominant allele containing a mutation in either \italicize{KRIT1}, \italicize{CCM2}, or \italicize{PDCD10}. Previous studies established that CCMs are caused by somatic mutations via a genetic two-hit mechanism similar to what I describe in Chapter 2. The two-hit mechanism tidily explains the variable penetrance of the inherited disorder and explains why sporadic cases have a single CCM whereas inherited cases have many; however, the two-hit mechanism does not account for the extreme variation in the severity of CCMs in the same individual, nor why some CCMs rapidly grow after years of quiescence. I hypothesized that some of the most aggressive CCMs may harbor somatic mutations in other genes that fuel CCM growth. To test this hypothesis, I sequenced 79 sporadic and inherited CCMs and found that 71\% of these CCMs all had a gain of function somatic mutation in the same gene: \italicize{PIK3CA}. Surprisingly, many of the CCMs had mutations in \italicize{PIK3CA} \italicize{and} either \italicize{KRIT1}, \italicize{CCM2}, or \italicize{PDCD10}; as many as three distinct somatic mutations within a single lesion. This finding prompted a new question: are all of the mutations present within the same cell, or are two different populations of cells merging to form a mosaic CCM? To answer this question, I collected nuclei from frozen CCM samples and sequenced the DNA from individual nuclei to determine the cellular phase of the somatic mutations. I sequenced a total of 21,221 nuclei across 2 inherited and 6 sporadic CCMs and all 8 confirmed that the somatic mutations are present within a single clonal population of cells. This study is the first to find that multiple genes contribute to the pathogenesis of a vascular malformation and highlights PIK3CA as a major contributor to CCM progression. In addition, the identification of \italicize{PIK3CA} mutations suggested rapamycin (an inhibitor of \italicize{PIK3CA} signaling) as a potential therapeutic for CCM and has since proved highly effective in preventing CCMs in mice. 

In Chapter 4, I find that Developmental Venous Anomalies (DVA) are caused by a somatic mutation in \italicize{PIK3CA} and predisposes to the formation of CCMs. The finding in Chapter 3 that some CCMs harbor as many as three distinct somatic mutations all within the same cell was somewhat surprising. Multiple mutations are common in cancer, but cancer develops in the context of uncontrolled growth and genomic instability. To understand how multiple mutations were occurring in CCMs we looked for signs of genomic instability in CCMs via sequencing and stains for DNA damage but found no evidence of elevated mutation rates. Instead, I found an answer in the form of DVA. DVA are the most common vascular malformation, present in up to 16\% of the population. DVA are considered to be harmless; however, almost all sporadic CCM form directly adjacent to a DVA. The association of DVA and CCM has been known for decades, but the cause has remained a mystery. I hypothesized that DVA could result from somatic \italicize{PIK3CA} mutation during developmental angiogenesis, creating a field of mutant cells that may develop into a CCM upon subsequent mutations. I tested this hypothesis using droplet digital PCR and found that \italicize{PIK3CA} mutation was present in both the CCM and the associated DVA, but that other pathogenic somatic mutations were found exclusively in the CCM. These data suggest that DVA are acting as a genetic primer that predisposes to CCM formation.

My results demonstrate that somatic mutations play an integral role in the development of VMs associated with HHT and CCM, and constitutes the first evidence of a VM that develops as a result of digenic somatic mutations. This opens new avenues of research into the mechanisms VM development, and identifies promising therapeutics that are currently being pursued for clinical trials. Before describing my results in detail, the remainder of this chapter provides a brief introduction to vascular malformations and previous research into HHT and CCM.





\section{Background}
***a bit about vascular malformations to flow into the next subsection***

\subsection{Hereditary Hemorrhagic Telangiectasia}

\subsubsection{Phenotypes}
\subsubsection{Associated Vascular Malformations}
\subsubsection{Genetics}
\subsubsection{TGF-$\beta$ and BMP Signaling}
\subsubsection{Animal Models}
\subsubsection{Sporadic Brain AVMs}

\subsection{Cerebral Cavernous Malformation}

\subsubsection{Phenotypes}
\subsubsection{Genetics}
\subsubsection{The CCM Complex}
\subsubsection{Animal Models}
\subsubsection{Developmental Venous Anomalies}





*****random sentences that might be useful later*****



There is strong genetic component to VMs and many inherited VM disorders that initially appear to contradict the focal nature of VMs. In people with an inherited VM disorder, the pathogenic mutation is present in every cell in their body. Why then, do such individuals develop a VM in their left hand and not their right hand? The answer is that additional local events occur after birth that seed the formation of a VM in a particular location; many such events in VMs take the form of somatic mutations. 

VMs may occur sporadically in otherwise healthy individuals, however there are also several familial mendelian disorders characterized by the development of VMs. Many of the familial VM disorders are inherited via an autosomal dominant allele. Despite the systemic presence of the causal mutation in these disorders, the resulting VMs are strictly focal lesions








In the past VMs were considered to be congenital lesions; while many VMs are present at birth, there are numerous examples of spontaneous \italicize{de novo} VM formation in adulthood. 



This observation likely reflects the fact that a systemic defect in vascular development is likely incompatible with life, therefore individuals with such severe vascular defects do not survive to birth. 



VMs may occur throughout the body and may affect various vascular beds (i.e.~capillaries, veins, and arteries). While VMs may be restricted to a single type of vessel, they are often observed as mixed vascular malformations (e.g.~arterio-venous malformation) and mixed vascular-lymphatic malformations (e.g.~capillary-lymphatic-venous malformation). 














